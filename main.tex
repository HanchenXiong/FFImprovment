%%%%%%%%%%%%%%%%%%%%%%%%%%%%%%%%%%%%%%%%%
% Beamer Presentation
% LaTeX Template
% Version 1.0 (10/11/12)
%
% This template has been downloaded from:
% http://www.LaTeXTemplates.com
%
% License:
% CC BY-NC-SA 3.0 (http://creativecommons.org/licenses/by-nc-sa/3.0/)
%
%%%%%%%%%%%%%%%%%%%%%%%%%%%%%%%%%%%%%%%%%

%----------------------------------------------------------------------------------------
%	PACKAGES AND THEMES
%----------------------------------------------------------------------------------------

\documentclass{beamer}

\mode<presentation> {

% The Beamer class comes with a number of default slide themes
% which change the colors and layouts of slides. Below this is a list
% of all the themes, uncomment each in turn to see what they look like.

%\usetheme{default}
%\usetheme{AnnArbor}
%\usetheme{Antibes}
%\usetheme{Bergen}
%\usetheme{Berkeley}
%\usetheme{Berlin}
%\usetheme{Boadilla}
\usetheme{CambridgeUS}
%\usetheme{Copenhagen}
%\usetheme{Darmstadt}
%\usetheme{Dresden}
%\usetheme{Frankfurt}
%\usetheme{Goettingen}
%\usetheme{Hannover}
%\usetheme{Ilmenau}
%\usetheme{JuanLesPins}
%\usetheme{Luebeck}
%\usetheme{Madrid}
%\usetheme{Malmoe}
%\usetheme{Marburg}
%\usetheme{Montpellier}
%\usetheme{PaloAlto}
%\usetheme{Pittsburgh}
%\usetheme{Rochester}
%\usetheme{Singapore}
%\usetheme{Szeged}
%\usetheme{Warsaw}

% As well as themes, the Beamer class has a number of color themes
% for any slide theme. Uncomment each of these in turn to see how it
% changes the colors of your current slide theme.

%\usecolortheme{albatross}
%\usecolortheme{beaver}
%\usecolortheme{beetle}
%\usecolortheme{crane}
%\usecolortheme{dolphin}
%\usecolortheme{dove}
%\usecolortheme{fly}
%\usecolortheme{lily}
%\usecolortheme{orchid}
%\usecolortheme{rose}
%\usecolortheme{seagull}
\usecolortheme{seahorse}
%\usecolortheme{whale}
%\usecolortheme{wolverine}

%\setbeamertemplate{footline} % To remove the footer line in all slides uncomment this line
%\setbeamertemplate{footline}[page number] % To replace the footer line in all slides with a simple slide count uncomment this line

%\setbeamertemplate{navigation symbols}{} % To remove the navigation symbols from the bottom of all slides uncomment this line
}

\usepackage{graphicx} % Allows including images
\usepackage{booktabs} % Allows the use of \toprule, \midrule and \bottomrule in tables
\usepackage{amsmath}
\usepackage{mathtools}

%----------------------------------------------------------------------------------------
%	TITLE PAGE
%----------------------------------------------------------------------------------------

\title[Online Decision-making]{Online Decision-making with a Expert Committee and Its Application on FahsionFlow} % The short title appears at the bottom of every slide, the full title is only on the title page

\author{Zalando Search Team} % Your name
\institute[Zalando SE] % Your institution as it will appear on the bottom of every slide, may be shorthand to save space
{
Zalando SE \\ % Your institution for the title page
\medskip
\textit{hanchen.xiong@zalando.de} % Your email address
}
\date{\today} % Date, can be changed to a custom date

\begin{document}

\begin{frame}
\titlepage % Print the title page as the first slide
\end{frame}

\begin{frame}
\frametitle{Overview} % Table of contents slide, comment this block out to remove it
\tableofcontents % Throughout your presentation, if you choose to use \section{} and \subsection{} commands, these will automatically be printed on this slide as an overview of your presentation
\end{frame}

%----------------------------------------------------------------------------------------
%	PRESENTATION SLIDES
%----------------------------------------------------------------------------------------

%------------------------------------------------
\section{Prediction with experts, game playing and portfolio management}        % Sections can be created in order to organize your presentation into discrete blocks, all sections and subsections are automatically printed in the table of contents as an overview of the talk
\subsection{Prediction with Experts' advice}
\frame{
    \frametitle{A gentle start} 
    \begin{itemize}
    \item<1->  \structure{Task:}  online prediction of a binary sequence, \emph{i.e.} sequentially forecast a value $y^{(t)}\in\{-1,1\}$ at time $t$ based on historical values $\{y^{(i)}\}_{i=1}^t$;
    \item<2->  \structure{Expert committee:} an expert $E_{n}$ is a man/woman who can make a prediction, $f_{n}^{(t)}$, using different strategies (algorithms/heuristics/data resources); assume there are $N$ experts in the committee; 
    \item<3->  \structure{Side information:} other task-relevant information may available, \emph{e.g.} $\{\mathbf{x}^{(t)}\}_{i=1}^{t}$;
    \item<4->  \structure{Forecaster: } an aggregating policy $\pi$ is a function which map experts' advices $\Longrightarrow$ final decision $\hat{y}_t$: $\hat{y}_t = \pi(f_{1}^{(t)}, f_{2}^{(t)}, \cdots, f_{N}^{(t)})$; 
    \item<5->  \structure{Weights update:} at the beginning, each expert is assigned a (uniform) weight $w_{n,0}=1$, and it will be updated along the sequence; 
    \item<6->  \structure{Loss:}  $l(\hat{y}^{(t)}, y^{(t)}) = \mathbf{1}_{\hat{y}^{(t)} \neq y^{(t)}}$, $l(f_{n}^{(t)}, y^{(t)}) = \mathbf{1}_{f_{n}^{(t)} \neq y^{(t)}}$
    \end{itemize}
}


\frame{
    \frametitle{A gentle start, cont.} 
    \onslide<+->{\begin{block}{A simple policy}
    The final decision is made with the \alert{majority voting} from the expert committee: $\hat{y}^{(t)} = \textbf{sign} (\frac{\sum_{n=1}^N f_{i}^{(t)}}{N} - 0.5 )$;
    \end{block}}
    
    \onslide<+->{\begin{block}{A simple weight update scheme}
    $w_{n}^{(t)} \leftarrow 0$ if expert $E_{n}$ makes a mistake at time $t-1$, \emph{i.e.} kick $E_n$ out of the committee; 
    \end{block}}

    \onslide<+->{\begin{block}{An ideal scenario}
    we know that there exist some experts who are perfect in the given task;  
    \end{block}}

    \onslide<+->{
    \textbf{Cummulative loss}:
    \begin{equation}
     \hat{L}^{(t)} = \sum_{i=t}^{t}l(\hat{y}^{(t)}, y^{(t)}) \leq \log_{2}N; 
    \end{equation}
    }
}

\frame{
    \frametitle{Generalized committee}
    \onslide<+->{\begin{block}{A more realistic scenario}
     we know that in the committee there exists a best expert who can work better than others in the given task;
    \end{block}}
    \onslide<+->{
    \textbf{Regret $R_{n}^{(t)}$}: the extra losses the forecaster made without exclusively following the expert $E_n$  up to time $t$: 
    \begin{equation}
     R_{n}^{(t)}= \hat{L}^{(t)} - L_{n}^{(t)} = \sum_{i=t}^{t}l(\hat{y}^{(t)}, y^{(t)}) - \sum_{i=t}^{t}l(f_{n}^{(t)}, y^{(t)})
    \end{equation}
    }

    \onslide<+->{\begin{block}{the upper bound of regret}
    $R^{(t)*} = \max_{n\in[1,N]} R_{n}^{(t)} = \hat{L}^{(t)} - \min_{n\in[1,N]}L_{n}^{(t)}$
    \end{block}}
}
    

\frame{
    \frametitle{Weighted majority algorithm}
    \begin{itemize}
    \item  \structure{Task:}  online prediction of a binary sequence, \emph{i.e.} sequentially forecast a value $y^{(t)}\in\{-1,1\}$ at time $t$ based on historical values $\{y^{(i)}\}_{i=1}^t$;
    \item  \structure{Expert committee:} an expert $E_{n}$ is a man/woman who can make a prediction, $f_{n}^{(t)}$, using different strategies (algorithms/heuristics/data resources); assume there are $N$ experts in the committee; 
    \item  \structure{Loss:}  $l(\hat{y}^{(t)}, y^{(t)}) = \mathbf{1}_{\hat{y}^{(t)} \neq y^{(t)}}$, $l(f_{n}^{(t)}, y^{(t)}) = \mathbf{1}_{f_{n}^{(t)} \neq y^{(t)}}$
    \line(1,0){450} 
    \item <2->  \alert{Forecaster: }  $\hat{y}_t = \textbf{sign}(\sum_{n:f_{n,t} =1}w_n^{(t-1)} - \sum_{m:f_{m,t} =-1}w_m^{(t-1)} )$;
    \item <3->  \alert{Weights update:} if one expert $E_n$ predicts wrongly, decrease its weight 
    \begin{equation}
       w_{n}^{(t+1)} = (1-\eta) w_n^{(t)}  
    \end{equation}
    where $\eta \leq \frac{1}{2}$. 
    \end{itemize}   
}


\frame{
    \frametitle{Analysis on weighted majority algorithm}
    \onslide<+->{
    \begin{theorem}[Regret bound using weighted majority]
    After T steps, $\hat{L}^{(T)} \leq 2(1+\eta) \min_{1\in[1,N]} L_{n}^{(T)} + \frac{2\ln N}{\eta}$
    \end{theorem}
    }
    \onslide<+->{
    \alert{Proof:}  Let $\Gamma^{(t)} = \sum_{n \in [1,N]} w_n^{(t)}$, then $\Gamma^{(1)}= N$. Also, if the forecaster makes 
    a mistake, $\hat{y}^{(t)} \neq y^{(t)}$ 
    \begin{equation}
      \Gamma^{(t+1)} \leq \Gamma^{(t)} (\frac{1}{2} + \frac{1}{2} (1-\eta) = \Gamma^{(t)} (1- \frac{\eta}{2})
    \end{equation}
    therefore: 
    \begin{equation}
       \Gamma^{(T+1)} \leq N (1-\frac{\eta}{2})^{\hat{L}^{(T)}} 
    \end{equation}
    for any individual expert $n$
    \begin{equation}
       w_n^{(T+1)}  = (1-\eta)^{L_n^{(T)}}
    \end{equation}
    since $w_n^{(T+1)} \leq \Gamma^{(T+1)} \Longrightarrow(1-\eta)^{L_n^{(T)}} \leq N (1-\frac{\eta}{2})^{\hat{L}^{(T)}} $
    }  
}

\frame{
    \frametitle{Analysis on weighted majority algorithm, cont.}
    \begin{equation}
    \begin{array}{rcl}
                        (1-\eta)^{L_n^{(T)}}    & \leq &  N (1-\frac{\eta}{2})^{\hat{L}^{(T)}}  \\ \\
    \Leftrightarrow         L_n^{(T)} \ln (1-\eta)  & \leq &  \ln N +  \hat{L}^{(T)}  \ln (1-\frac{\eta}{2}) \\ \\
    \Leftrightarrow       -\ln (1-\frac{\eta}{2})   & \leq &  -L_n^{(T)} \ln (1-\eta) + \ln N  \\ \\  
    \xRightarrow{\alert{x\leq -\ln (1-x)}}  \frac{\eta}{2} \hat{L}^{(T)} & \leq & -L_n^{(T)} \ln (1-\eta) + \ln N  \\ \\  
    \xRightarrow{\alert{-\ln (1-x) \leq x + x^2, \text{when } x \leq 1/2 }}  \frac{\eta}{2}  \hat{L}^{(T)} & \leq & L_n^{(T)} \eta(1+\eta) + \ln N  \\ \\  
    \Leftrightarrow  \hat{L}^{(T)} & \leq & 2(1+\eta) L_{n}^{(T)} + \frac{2\ln N}{\eta}
    \end{array}
    \end{equation} 
}

\frame{
    \frametitle{Randomized weighted majority algorithm}
    \begin{itemize}
    \item  \structure{Task:}  online prediction of a binary sequence, \emph{i.e.} sequentially forecast a value $y^{(t)}\in\{-1,1\}$ at time $t$ based on historical values $\{y^{(i)}\}_{i=1}^t$;
    \item  \structure{Expert committee:} an expert $E_{n}$ is a man/woman who can make a prediction, $f_{n}^{(t)}$, using different strategies (algorithms/heuristics/data resources); assume there are $N$ experts in the committee; 
    \item  \structure{Loss:}  $l(\hat{y}^{(t)}, y^{(t)}) = \mathbf{1}_{\hat{y}^{(t)} \neq y^{(t)}}$, $l(f_{n}^{(t)}, y^{(t)}) = \mathbf{1}_{f_{n}^{(t)} \neq y^{(t)}}$
    \item  \structure{Weights update:} if one expert $E_n$ predicts wrongly, decrease its weight 
    \begin{equation}
       w_{n}^{(t+1)} = (1-\eta) w_n^{(t)}  
    \end{equation}
    where $\eta \leq 1/2$. 
    \line(1,0){450} 
    \item <2->  \alert{Forecaster: }  $\hat{y}_t \sim  \textbf{Bernoulli}\left(\frac{\sum_{n:f_{n,t} =1} w_n^{(t-1)}}{\sum_n w_n^{(t-1)}}, \frac{\sum_{n:f_{n,t} =-1} w_n^{(t-1)}}{\sum_n w_n^{(t-1)}}\right)$;
    \end{itemize}   
}

\frame{
    \frametitle{Analysis on randomized weighted majority algorithm}
    \onslide<+->{
    \begin{theorem}[Regret bound using randomized weighted majority]
    After T steps, $\hat{L}^{(T)} \leq (1+\eta) \min_{1\in[1,N]} L_{n}^{(T)} + \frac{\ln N}{\eta}$
    \end{theorem}
    }
    \onslide<+->{
    \alert{Proof:}  Let $\Gamma^{(t)} = \sum_{n \in [1,N]} w_n^{(t)}$, then $\Gamma^{(1)}= N$. 
    \newline 
    At each time $t$, let $F^(t) = \frac{\sum_{n:f_n^{(t)} \neq y^{(t)}} w^{(t)}_n }{\sum_{n} w_n^{(t)}}$, then 
    \begin{equation}
      \Gamma^{(t+1)} = \Gamma^{(t)} \Big( 1-F^{(t)} + F^{(t)}(1-\eta) \Big) = \Gamma^{(t)} (1- F^{(t)}\eta)
    \end{equation}
    therefore: 
    \begin{equation}
       \Gamma^{(T+1)} = N \prod_{t=1}^T(1-F^{(t)}\eta) 
    \end{equation}
    for any individual expert $n:       w_n^{(T+1)}  = (1-\eta)^{L_n^{(T)}}$, \\
    since $w_n^{(T+1)} \leq \Gamma^{(T+1)} \Longrightarrow(1-\eta)^{L_n^{(T)}} \leq N \prod_{t=1}^T(1-F^{(t)}\eta)$
    }  
}


\frame{
    \frametitle{Go beyond 0-1 loss}
    \begin{itemize} 
    \item  \structure{Task:}  online prediction of a sequence, \emph{i.e.} sequentially forecast a value $y^{(t)}\in\{-1,1\}$ at time $t$ based on historical values $\{y^{(i)}\}_{i=1}^t$;
    \item  \structure{Expert committee:} an expert $E_{n}$ is a man/woman who can make a prediction, $f_{n}^{(t)}$, using different strategies (algorithms/heuristics/data resources); assume there are $N$ experts in the committee; 
    \item  \structure{Loss:}  $l(\hat{y}^{(t)}, y^{(t)}) = \mathbf{1}_{\hat{y}^{(t)} \neq y^{(t)}}$, $l(f_{n}^{(t)}, y^{(t)}) = \mathbf{1}_{f_{n}^{(t)} \neq y^{(t)}}$
    \item  \structure{Weights update:} if one expert $E_n$ predicts wrongly, decrease its weight 
    \begin{equation}
       w_{n}^{(t+1)} = (1-\eta) w_n^{(t)}  
    \end{equation}
    where $\eta \leq 1/2$. 
    \item <2-> \alert{Forecaster: }  $\hat{y}_t \sim  \textbf{Bernoulli}\left(\frac{\sum_{n:f_{n,t} =1} w_n^{(t-1)}}{\sum_n w_n^{(t-1)}}, \frac{\sum_{n:f_{n,t} =-1} w_n^{(t-1)}}{\sum_n w_n^{(t-1)}}\right)$;
    \end{itemize}   
}


\frame{
    \frametitle{Analysis on randomized weighted majority algorithm, cont.}
    \begin{equation}
    \begin{array}{rcl}
                        (1-\eta)^{L_n^{(T)}}    & \leq &  N \prod_{t=1}^T(1-F^{(t)}\eta) \\ \\
    \Leftrightarrow         L_n^{(T)} \ln (1-\eta)  & \leq & \ln N + \sum_{t=1}^T \ln(1-F^{(t)}\eta) \\ \\
    \Leftrightarrow        - \sum_{t=1}^T \ln (1-F^{(t)}\eta)   & \leq &  -L_n^{(T)} \ln (1-\eta) + \ln N  \\ \\  
    \xRightarrow{\alert{x\leq -\ln (1-x)}} \eta \underbrace{\sum_{t=1}^T F^{(t)}}_{\mathbb{E} \{\hat{L}^{(T)}\}\approx \hat{L}^{(T)} } & \leq & -L_n^{(T)} \ln (1-\eta) + \ln N  \\ \\  
    \xRightarrow{\alert{-\ln (1-x) \leq x + x^2, \text{when } x \leq 1/2 }}  \eta \hat{L}^{(T)} & < & L_n^{(T)} \eta(1+\eta) + \ln N  \\ \\  
    \Leftrightarrow  \hat{L}^{(T)} & \leq & (1+\eta) L_{n}^{(T)} + \frac{\ln N}{\eta}
    \end{array}
    \end{equation} 
}


\subsection{Online repeated game playing}
\frame{
 \frametitle{One-player game}
 \begin{itemize}
 \item<1-> One player plays a game, at time $t$ he takes one action $a^{(t)} = i\in[1,N]$, then the environment 
 releases the cost for each action $\mathbf{m}^{(t)} = [m_1^{(t)}, m_2^{(t)},\cdots, m_N^{(t)}]^\top$,
 \item<2-> \alert{Note} that the loss function $\mathbf{m}^{(t)}$ can change over time, i.e. the environment is changing.
 \item<3-> the player takes actions with some  
 \end{itemize}
}

\frame{
\frametitle{Two-player game} 
\alert{Two-person zero-sum}: two players ($K=2)$ play a game which is defined by a $N\times N$ cost matrix $\mathbf{C}$ ($N$ is 
the number of possible actions for players) , where 
each entry $c_{ij}$ defines \alert{the loss to the row player} when the \structure{row player} takes the action $i\in [1,N]$ and 
the \structure{column player} takes the action $j\in [1,N]$. 


An example cost matrix: 
$\left[\begin{array}{ccccc}
  &   1         &   2         &   3         &   4   \\
1 &   c_{11}    &   c_{12}    &   c_{13}    &   c_{14} \\
2 &   c_{21}    &   c_{22}    &   c_{23}    &   c_{24} \\
3 &   c_{31}    &   c_{32}    &   c_{33}    &   c_{34} \\
4 &   c_{41}    &   c_{42}    &   c_{43}    &   c_{44} 
\end{array}
\right]$

The row player's goal is to minimize its loss while the objective of the column player is to maximize it 

}

\frame{
    \frametitle{Nash Equilibrium}
    \begin{itemize}
      \item<1-> no player has an incentive of changing his strategy if the player does not change his, i.e. every player is happy about 
      current status; 
      \item<2-> 
    \end{itemize}
}


\frame{
    \frametitle{Hanan's theorem}
    follow the perturbed leading expert; 
}

\frame{
    follow the perturbed leading expert; 
}

\subsection{Universal portfolio}
\frame{
    \frametitle{Nonlinear loss}
    $f_t{i} = \log (-\langle \mathbf{p}_i, \Delta x_{t} \rangle)$
}

\frame{
    \frametitle{Han}
}

%-----------------------------------------------------------------------------------------
\section{Bandit Optimization in metric spaces}
\subsection{Bandit: play games with limited feedbacks}
\subsection{Gaussian process bandit optimization}
\subsection{General Bayesian optimization} 

%------------------------------------------------------------------------------------------
\section{Concept drift in online decision}
\subsection{Stability v.s. adaptivity}
\subsection{Explicite detection of concept drifts}
\subsection{Adaptive regret for tracking the best expert}
\subsection{Time-varing suface-responce bandit optimization}



%------------------------------------------------------------------------------------------
\section{Sku Proposal in FashionFlow}
\subsection{Expert setting v.s. Bandit setting}
\subsection{From a good classifier to a good proposer}

%------------------------------------------------

\begin{frame}
\frametitle{Paragraphs of Text}
Sed iaculis dapibus gravida. Morbi sed tortor erat, nec interdum arcu. Sed id lorem lectus. Quisque viverra augue id sem ornare non aliquam nibh tristique. Aenean in ligula nisl. Nulla sed tellus ipsum. Donec vestibulum ligula non lorem vulputate fermentum accumsan neque mollis.\\~\\

Sed diam enim, sagittis nec condimentum sit amet, ullamcorper sit amet libero. Aliquam vel dui orci, a porta odio. Nullam id suscipit ipsum. Aenean lobortis commodo sem, ut commodo leo gravida vitae. Pellentesque vehicula ante iaculis arcu pretium rutrum eget sit amet purus. Integer ornare nulla quis neque ultrices lobortis. Vestibulum ultrices tincidunt libero, quis commodo erat ullamcorper id.
\end{frame}

%------------------------------------------------

\begin{frame}
\frametitle{Bullet Points}
\begin{itemize}
\item Lorem ipsum dolor sit amet, consectetur adipiscing elit
\item Aliquam blandit faucibus nisi, sit amet dapibus enim tempus eu
\item Nulla commodo, erat quis gravida posuere, elit lacus lobortis est, quis porttitor odio mauris at libero
\item Nam cursus est eget velit posuere pellentesque
\item Vestibulum faucibus velit a augue condimentum quis convallis nulla gravida
\end{itemize}
\end{frame}

%------------------------------------------------

\begin{frame}
\frametitle{Blocks of Highlighted Text}
\begin{block}{Block 1}
Lorem ipsum dolor sit amet, consectetur adipiscing elit. Integer lectus nisl, ultricies in feugiat rutrum, porttitor sit amet augue. Aliquam ut tortor mauris. Sed volutpat ante purus, quis accumsan dolor.
\end{block}

\begin{block}{Block 2}
Pellentesque sed tellus purus. Class aptent taciti sociosqu ad litora torquent per conubia nostra, per inceptos himenaeos. Vestibulum quis magna at risus dictum tempor eu vitae velit.
\end{block}

\begin{block}{Block 3}
Suspendisse tincidunt sagittis gravida. Curabitur condimentum, enim sed venenatis rutrum, ipsum neque consectetur orci, sed blandit justo nisi ac lacus.
\end{block}
\end{frame}

%------------------------------------------------

\begin{frame}
\frametitle{Multiple Columns}
\begin{columns}[c] % The "c" option specifies centered vertical alignment while the "t" option is used for top vertical alignment

\column{.45\textwidth} % Left column and width
\textbf{Heading}
\begin{enumerate}
\item Statement
\item Explanation
\item Example
\end{enumerate}

\column{.5\textwidth} % Right column and width
Lorem ipsum dolor sit amet, consectetur adipiscing elit. Integer lectus nisl, ultricies in feugiat rutrum, porttitor sit amet augue. Aliquam ut tortor mauris. Sed volutpat ante purus, quis accumsan dolor.

\end{columns}
\end{frame}



\begin{frame}
\frametitle{Table}
\begin{table}
\begin{tabular}{l l l}
\toprule
\textbf{Treatments} & \textbf{Response 1} & \textbf{Response 2}\\
\midrule
Treatment 1 & 0.0003262 & 0.562 \\
Treatment 2 & 0.0015681 & 0.910 \\
Treatment 3 & 0.0009271 & 0.296 \\
\bottomrule
\end{tabular}
\caption{Table caption}
\end{table}
\end{frame}

%------------------------------------------------

\begin{frame}
\frametitle{Theorem}
\begin{theorem}[Mass--energy equivalence]
$E = mc^2$
\end{theorem}
\end{frame}

%------------------------------------------------

\begin{frame}[fragile] % Need to use the fragile option when verbatim is used in the slide
\frametitle{Verbatim}
\begin{example}[Theorem Slide Code]
\begin{verbatim}
\begin{frame}
\frametitle{Theorem}
\begin{theorem}[Mass--energy equivalence]
$E = mc^2$
\end{theorem}
\end{frame}\end{verbatim}
\end{example}
\end{frame}

%------------------------------------------------

\begin{frame}
\frametitle{Figure}
Uncomment the code on this slide to include your own image from the same directory as the template .TeX file.
%\begin{figure}
%\includegraphics[width=0.8\linewidth]{test}
%\end{figure}
\end{frame}

%------------------------------------------------

\begin{frame}[fragile] % Need to use the fragile option when verbatim is used in the slide
\frametitle{Citation}
An example of the \verb|\cite| command to cite within the presentation:\\~

This statement requires citation \cite{p1}.
\end{frame}

%------------------------------------------------

\begin{frame}
\frametitle{References}
\footnotesize{
\begin{thebibliography}{99} % Beamer does not support BibTeX so references must be inserted manually as below
\bibitem[Smith, 2012]{p1} John Smith (2012)
\newblock Title of the publication
\newblock \emph{Journal Name} 12(3), 45 -- 678.
\end{thebibliography}
}
\end{frame}

%------------------------------------------------

\begin{frame}
\Huge{\centerline{The End}}
\end{frame}

%----------------------------------------------------------------------------------------

\end{document} 
